\documentclass[a4paper,12pt]{jsarticle}
\usepackage{framed}

\begin{document}
\begin{framed}
  \section*{要旨}
  近年、小学校で「英語」が正式な科目として認定されたように、グローバル人材の需要は拡大している。
  しかし、世界基準の英語力テストの結果を見てみると、日本人の英語力は決して高いとは言えず、特にこれまでの英語教育で力を入れられていなかった「スピーキング力」が弱点である。
  そこで本研究では、ユーザーのスピーキング力を向上させることを目的に、深層学習を使ってユーザーの発言に対して深堀り質問を生成する対話モデルを構築する。
  
  \section*{研究の背景と目的}
  英語力は、Reading/Listening/Writing/Speakingの大きく4技能に分類される。
  この4技能全ての英語力を測る試験であるTOEFLの結果では、日本は世界163カ国中135位、アジア28カ国中26位となっており、高い成績とは言えない結果である。
  また、技能別にスコアを見てみると、WritingやSpeakingのようなアウトプットを要する技能の得点がReadingやListeningと比較して低くなっている特徴がある。
  日本人の英語力が低いことの1つの要因として、普段から英語を発言する機会が圧倒的に少ないことが挙げられる。
  例えば、TOEFLの結果がアジア28カ国中5位のフィリピンでは、幼稚園から英語以外の数学や理科のような科目も英語を使った授業が行われており、日常的に英語に触れている。
  そのため、自ら発言を行う機会も多く得られるため、Speaking力を含めた英語力全体のスコアが高くなっている。
  それに対して、これまでの日本の英語教育はReadingに力を入れており、自らの意見を発する機会はほとんど無い。
  英語のアウトプットをすることを目的とした学習サービスはすでに多数存在し、大きくアプリと対人での学習サービスがある。

  アプリの課題として、あらかじめ決められた会話の流れに沿って進められるため、アウトプットよりもインプットに近い学習スタイルになっていることが挙げられる。
  つまり、日常会話のように瞬時に次の会話文を自分で生成しているわけではなく、予め用意された台本を読み上げる学習スタイルが大半である。

  また、対人サービスの課題として、担当する先生によって学習の質が大きく変わってしまうことが挙げられる。
  先生の英語レベルやコミュニケーション力、また生徒との相性などに応じて、同じ学習時間で得られる質は異なる。
  既存のオンライン英会話サービスで、評価の高い先生と評価の低い先生に付けられたレビューを調査したところ、
  評価の高い先生には「ユーザーの話を聞き出すような質問をしてくれるため、話しやすい」といった声が多く見つかり、
  評価の低い先生には「先生自身が話してばかりのため、ユーザー自身のアウトプットをする機会が少ない」という声が多く見つかった。

  そこで本研究では、先生の実力に応じてアウトプットの数が変わってしまう課題に取り組む。
  機械学習を用いて、ユーザーの発言に対して「よりユーザーが話しやすくなるような」質問を自動で生成するシステムの開発を目指す。
  これにより、ユーザーは自分の頭で構築した英文をアウトプットする機会が増え、英語力全体の向上を促すことが本研究の目的である。

  \section*{研究の特色とインパクト}
  本研究の特色は、構築する会話モデルが深堀り質問のみを生成する点である。
  一般的な会話生成では、入力された文章に沿った内容の文章を生成することが目的である。
  一方、本研究ではユーザーの発言を促すことが目的であるため、システム自身の話をするのではなく、よりユーザーが話しやすくなるような文章を生成する。
  会話の文脈に応じた質問生成に特化した研究はこれまでに行われていないため、会話データセットも存在せず、本研究で初めて作成する。

  \section*{これまでの成果}
  はじめに質問生成を行うための会話データセットを作成した。
  データは以下3つの条件で絞った会話をTwitterから収集した。
  1. 2ターン以上やり取りがあるもの(A→B→A→B)
  2. 5W1Hで始まり「?」で会話が終了している
  3. ネットスラングを含まない
  次に、会話の文脈に応じた質問を生成を行う深層学習モデルの構築をした。
  具体的には、GPTと呼ばれる言語モデルに会話の文脈を受け取れるように学習させたDialoGPTである。
  作成したデータセットを用いてモデルを学習し、入力した会話文から質問を生成させる実験を行った結果、質問生成確立は56.1%となった。
  これは、GPTからDigloGPTへの学習時に非質問生成文が含まれていることが要因として考えられる。
  しかし、
  \end{framed}

  \newpage
  \begin{framed}
  \section*{研究計画}
  どんな形でこれから研究を進めるかを書いていく
  \subsection*{1.モデルの構築}
  特にモデルについて
  \subsection*{2.会話質問生成}
  どんな文章が生成されるのか?
  \subsection*{3.モデルの評価実験}
  評価軸について
  \end{framed}

\end{document}